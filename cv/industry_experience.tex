\begin{industryExperience}
    \begin{enumerate}[leftmargin=0.45cm, itemsep=1em, topsep=0.5em, parsep=0.2em]
        \item
        \position{Assistant Professor (UK Lecturer, Teaching)}
        \employer{University College London}
        \dates{September 2024 to \textbf{present}}
        \place{London, UK}
        \jobdescription{I teach the Data Acquisition and Processing systems within the Integrated Machine Learning Master at UCL.
        The module is an introduction to data covering: data acquisition from web API and sensors; data storage, SQL and nosQL databases; statistical data processing including sampling, normalisation, and linear projections; deep learning.
        I supervise postgraduate students in their final projects and I am responsible to coordinate the assignments of projects across academics in the department.
        Finally, I tutor undergraduates to kickstart their journey in AI and machine learning.}
        % Skills: Teaching, supervision.

        \item
        \position{Research Assistant}
        \employer{Imperial College London}
        \dates{September 2019 to November 2020}
        \place{London, UK}
        \jobdescription{I conducted research to shorten the computational time for predictive modelling of surgical interventions in cardiology. Statistical modelling, specifically deep learning, is central in the approach. The deep networks take advantage of accepted numerical modelling techniques to generate training data and are optimised to infer approximate solutions in about 1\% of time necessary to standard models. The position was founded by the Rosetrees Trust, in collaboration with the ElectroCardioMaths program, a multidisciplinary initiative that brings together the National Heart and Lung Institute, and the Departments of Bioengineering, Aeronautics, Computing and Physics to address key challenges in the diagnosis and treatment of complex cardiac arrhythmia.}
        % Skills: Python, Jax, Flax, Pytorch, GPU computing, Git, Numerical methods, Predictive modelling.

        \item
        \position{Machine Learning and Decision Analytics Lead}
        \employer{BuroHappold Engineering}
        \dates{August 2018 to October 2020}
        \place{London, UK}
        \jobdescription{I led the applied research in Machine Learning to help understand how AI can create value for the business. We democratised the access to Deep Learning technologies to allow every employee to access the knowledge and the tools. We exploited Visual Programming, a recognised and diffused tool for design, to create a framework that interoperates between the most common deep learning libraries, tensorflow, keras, pytorch, numpy. The position was part of the wider computational core team that brings together discipline leads into a centralised research team bhom.xyz. The BHoM is currently adopted in different companies and is at the base of the project funded by Innovate UK, github.com/aecdeltas, which I advise for.}
        % Skills: Line management, Risk management, Planning, C\#, Python, Deep learning, NoSql, Keras, Pytorch, jupyterhub.

        \item
        \position{Computational/Machine Learning Engineer}
        \employer{BuroHappold Engineering}
        \dates{August 2017 to August 2018}
        \place{London, UK}
        \jobdescription{For 50\% of my time, I have been applying deep learning for computer vision to the analysis of security footage for the Premier League. To monitor the number of standing fans during a football match we created a database of more than 400,000 annotated images and trained a convolutional deep network to identify them. We then exploited principal component analysis, hierarchical clustering and bespoke data visualisation to gather insights from the resulting probability distribution.
        For the remaining 50\% of my time, I have been designing a framework for data sharing and co-creation in design, for the architecture, engineering and construction industry, bhom.xyz. We created a software-agnostic model to link together the capabilities of existing software and allow seamless interoperability between them. A short-cycles, distributed scrum development model, and an entity-component-system architecture allowed independent contributions from more than 50 users. My main responsibility has been to lead the UI and support the framework leadership of the project.}
        % Skills: Python, Supervised learning, Unsupervised learning, NoSql, Tensorflow, Pytorch, Azure, Cloud computing, Docker, C\#, JavaScript, Git, Agile development, SCRUM, UI, UX, MongoDB.

        \item
        \position{Intern Computational Engineer}
        \employer{BuroHappold Engineering}
        \dates{April 2017 to August 2017}
        \place{Bath, UK}
        \jobdescription{I provided computational support for the Stadia Atmosphere project and helped introduce deep learning into the current offer for sports venues design. We used pre-trained deep neural networks for Natural Language Processing to perform sentiment analysis on news regarding a specific football team.}
        % Skills: Python, Visual programming, C\#, MongoDB.

        \item
        \position{Architect}
        \employer{Gianni Ranaulo Design}
        \dates{September 2016 to December 2016.}
        \place{Dubai, UAE}
        \jobdescription{I provided support for the parametric modelling of a façade in a multi-purpose shopping centre.}
        % Skills: Visual Programming, Geometrical modelling, Image processing, Rendering.

        \item
        \position{Computational Architect}
        \employer{Gridshell.it}
        \dates{September 2015 to September 2016}
        \place{Naples, Italy}
        \jobdescription{I conducted research on the application of computational tools to recover the use of traditional low-tech construction techniques. I used generative modelling to provide cost-effective, environmentally efficient, and functionally viable structure. Using genetic algorithms, particle-spring system models and dynamic relaxation we designed and built 13 prototypes of timber post-formed gridshells. Taking advantage of recognised acoustic modelling techniques, we generatively designed three temporary acoustic shells for outdoor classical concerts, the last of which has won the Peter Lord Award.}
        % Skills: Python, Numerical methods, Physics, Differential Equations, Spectral analysis, Acoustical modelling.

        \item
        \position{Computational Architectural Assistant}
        \employer{Gridshell.it}
        \dates{July 2014 to September 2015}
        \place{Naples, Italy}
        \jobdescription{}

        \item
        \position{Intern Architect}
        \employer{CRC – Constructions Restorations and Consolidations}
        \dates{September 2012 to December 2012}
        \place{Naples, Italy}
        \jobdescription{I provided support for the preparation of compliance documentation for a multi-storey parking building. My main responsibility was to ensure the fire compliance of the building.}
    \end{enumerate}
\end{industryExperience}