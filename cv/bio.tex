Eduardo Pignatelli is Machine Learning Lead at BuroHappold Engineering and Research Assistant at Department of Bioengineering of the Imperial College London.

In 2014 he defended the thesis “Computational morphogenesis and construction of an acoustic shell for outdoor chamber music”. Implementing numerical methods that uses geometrical acoustics, computational physics, descriptive geometry and genetic algorithms, the work established the state of the art for generative method of passive acoustic shell for outdoor classical music.

In 2017 he joined BuroHappold Engineering, where he is now leading the applied research in Machine Learning to help understand how AI can create value for the business. He is pursuing the democratisation of the access to Deep Learning technologies to allow every employee to access the knowledge and the tools. Using Visual Programming, a recognised and diffused tool for design, to create a framework that interoperates between the most common deep learning libraries, tensorflow, keras, pytorch, numpy.

In 2019 he joined the Biologically-Inspired Computation and Inference Lab at the Imperial College. He is conducting research to shorten the computational time for predictive modelling of surgical interventions in cardiology, by exploring the use of deep learning to create surrogate models. The deep networks take advantage of accepted numerical modelling techniques to generate training data, and are optimised to infer approximate solutions in about 1\% of time necessary to standard models. The position is founded by the Rosetrees Trust, in collaboration with the ElectroCardioMaths program, a multidisciplinary initiative that brings together the National Heart and Lung Institute, and the Departments of Bioengineering, Aeronautics, Computing and Physics to address key challenges in the diagnosis and treatment of complex cardiac arrhythmias.


Short bio:

Eduardo is Machine Learning Lead at BuroHappold Engineering and Research Assistant in AI at the Imperial College London. He is an architect by training and a computer scientist by profession. Eduardo's interests are in Artificial General Intelligence and include deep learning, reinforcement learning, multi-agent systems, game theory and computer vision. His trajectory links together social sciences and the mathematical sciences. Eduardo is very experienced in computer science - check his LinkedIn https://www.linkedin.com/in/eduardo-pignatelli/ - skilled in engineering for research - check his GitHub https://github.com/epignatelli - and dedicated to the cause of discovery as demonstrated through publishing whilst working.


I investigate Deep Reinforcement Learning with a focus on the Credit Assignment Problem at UCL, as a PhD student under the supervision of Laura Toni and Tim Rocktäschel.
Previously, I was Machine Learning Lead at BuroHappold Engineering, where I focused on integrating machine automation into the design process.
I was a Research Assistant at the Department of Bioengineering of the Imperial College London, where I developed a surrogate model for the electrophysiology of the heart, to prevent cardiac arrhythmias.
I hold a Master of Science in Architecture from the University of Naples Federico II, where I developed generative models for the design of acoustic shells for outdoor chamber music using genetic algorithms.